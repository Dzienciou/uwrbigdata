\documentclass[12pt]{uebung}


\dozent{Przemysław Uznański}
\vorlesung{Algorithms for Big Data}
\semester{Fall Semester 2019}
%\tutoren{Tutoren name}
 
\usepackage{amsmath}
\usepackage{mathtools}
 
 
%\usepackage[utf8]{inputenc}
\usepackage{amsmath, amssymb,wasysym}
%\usepackage{tikz}
\newtheorem{definition}{Definition}
\newtheorem{theorem}{Theorem}

 
%\usepackage{multirow}

\usepackage[english]{babel}

 \begin{document}

 \startnummer{1}
 
\kopf[0]{04/11/2019}{5}


\renewcommand{\aufgname}{Exercise}

\begin{aufg}
Show how to use \textsf{CountMin} to support \emph{range queries}: $\textsf{range}(a,b) = \sum_{i=a}^b x_i \pm \varepsilon |x|_1$.
The space usage should be $\textrm{poly} \log$ worse than in the vanilla \textsf{CountMin}.\\
\textbf{Hint:} You can use a few \textsf{CountMin} structures to obtain datastructure with error guarantee slightly worse than in \textsf{CountMin}. You can later just take $\epsilon'$ to be a little smaller than $\varepsilon$ to offset this.
\end{aufg}

\begin{aufg}[Quantiles]
$\phi$-quantile of a multiset of size $n$ is $\phi \cdot n$-smallest element. $\varepsilon$-approximate $\phi$-quantile is any element that is between $(\phi-\varepsilon)$-quantile and $(\phi+\varepsilon)$-quantile. Use previous exercise to build a sketch that allows a queries of form $\textrm{quantile}(\phi)$ for a fixed in advance $\varepsilon$.
\end{aufg}


\begin{aufg}[Sketching for inner product, 2pts]
Let $X$ be a \textsf{CountMin} sketch of vector $x$, and $Y$ be a \textsf{CountMin} sketch of vector $y$, with $x$ and $y$ being non-negative. Both sketches are obtained used the same hashing. Our goal is to approximate $x \odot y = \sum_i x_i y_i$. Show that $\min_j \sum_k X[j][i] \cdot Y[j][i]$ estimates $x \odot y$ up to $\pm \varepsilon |x|_1 |y|_1$ error.
\end{aufg}

\end{document}
