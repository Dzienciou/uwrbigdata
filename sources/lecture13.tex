\documentclass[11pt]{article}
\usepackage{latexsym}
\usepackage{amsmath,amssymb,amsthm}
\usepackage{epsfig}
\usepackage[right=0.8in, top=1in, bottom=1.2in, left=0.8in]{geometry}
\usepackage{setspace}
\usepackage[utf8]{inputenc}
%\spacing{1.06}

\newcommand{\handout}[5]{
  \noindent
  \begin{center}
  \framebox{
    \vbox{\vspace{0.25cm}
      \hbox to 5.78in { {University of Wrocław:\hspace{0.12cm}Algorithms for
          Big Data (Fall'19)} \hfill #2 }
      \vspace{0.48cm}
      \hbox to 5.78in { {\Large \hfill #5  \hfill} }
      \vspace{0.42cm}
      \hbox to 5.78in { {#3 \hfill #4} }\vspace{0.25cm}
    }
  }
  \end{center}
  \vspace*{4mm}
}

\newcommand{\lecture}[4]{\handout{#1}{#2}{#3}{Scribe:\hspace{0.08cm}#4}{Lecture #1}}

\newtheorem{theorem}{Theorem}
\newtheorem{corollary}[theorem]{Corollary}
\newtheorem{lemma}[theorem]{Lemma}
\newtheorem{observation}[theorem]{Observation}
\newtheorem{example}[theorem]{Example}
\newtheorem{definition}[theorem]{Definition}
\newtheorem{claim}[theorem]{Claim}
\newtheorem{fact}[theorem]{Fact}
\newtheorem{property}[theorem]{Property}
\newtheorem{assumption}[theorem]{Assumption}

\newcommand{\E}{{\mathbb E}}
\DeclareMathOperator{\var}{Var}
\newcommand{\eps}{\varepsilon}
\newcommand{\bigo}{\mathcal{O}}
\setcounter{MaxMatrixCols}{20}

\begin{document}

\lecture{13: MPC}{ 20/01/2020}{Lecturer: \emph{Przemysław Uznański }}{-}

\section{Massively Parallel Computing}
Modern distributed computing model (captures map-reduce, hadoop, spark, etc.).

\begin{itemize}
\item input of size $n$
\item $M$ machines, each of space $S$, $S = n^{1-\delta}$, $M \cdot S = \bigo(n)$.
\item output (might be too large, e.g. size $n$ that does not fit on a single machine)
\end{itemize}

Computation:
\begin{itemize}
\item computation happens over $R$ rounds
\item each machine: near linear computation per round, so total computation cost $\bigo(n^{1+o(1)} R)$
\item each machine communicates $\sim S$ bits per round, so total communication cost $\bigo(n R)$
\end{itemize}
goal: minimize $R$

\subsection{Sorting (Tera-Sort)}
Intuition: if we partition input onto machines, so each machine receives contiguous fragment of size $S$, then we are done in a single round (each machine sorts and outputs its own part, output == concatenation of outputs).

Idea: 
\begin{itemize}
\item each machine receives input
\item each machine samples randomly from its input
\item each sample is sent to single machine ($1$ round)
\item 1 machine gathers all the samples, sorts locally, and sents back to everyone approximate histogram
\item machines use approximate histogram to decide how to partition locally their input and sent it to proper receivers
\item then everyone sorts their parts
\end{itemize}

Total sample size $k = \bigo(S)$, and whp histogram is with $\pm \varepsilon n$ error, where $\frac{\log n}{\varepsilon^2} \le k$. We are ok if error is $\bigo(S)$. This is satisfied when $S^{3/2} = \Omega(n \sqrt{\log n})$, or $S = \Omega(n^{2/3} (\log n)^{1/3})$ (thus whp we can sort in $\bigo(1)$ rounds).
\end{document}


