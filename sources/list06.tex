\documentclass[12pt]{uebung}


\dozent{Przemysław Uznański}
\vorlesung{Algorithms for Big Data}
\semester{Fall Semester 2019}
%\tutoren{Tutoren name}
 
\usepackage{amsmath}
\usepackage{mathtools}
 
 
%\usepackage[utf8]{inputenc}
\usepackage{amsmath, amssymb,wasysym}
%\usepackage{tikz}
\newtheorem{definition}{Definition}
\newtheorem{theorem}{Theorem}

 
%\usepackage{multirow}

\selectlanguage{USenglish}

 \begin{document}

 \startnummer{1}
 
\kopf[0]{18/11/2019}{6}

\newcommand{\bigo}{\mathcal{O}}
\renewcommand{\aufgname}{Exercise}

\begin{definition}[Hadamard matrix]
We define $H_1 = \begin{bmatrix} 1 \end{bmatrix}$ and $H_{2n} = \begin{bmatrix} H_n & H_n \\ H_n & -H_n \end{bmatrix}$. We will write $F = \frac{1}{\sqrt{n}} H_n$, dropping $n$ from the index (and assuming $n$ is a power of two).
\end{definition}

\begin{aufg}
Show that $\|F x\|_2 = \|x\|_2$ for any $x \in \mathbb{R}^n$.
\end{aufg}

\begin{aufg}
Show that $F \times F = I$.
\end{aufg}

\begin{aufg}
Show algorithm that given $x \in \mathbb{R}^n$ computes $Fx$ in time $\bigo(n \log n)$.
\end{aufg}


\end{document}