\documentclass[11pt]{article}
\usepackage{latexsym}
\usepackage{amsmath,amssymb,amsthm}
\usepackage{epsfig}
\usepackage[right=0.8in, top=1in, bottom=1.2in, left=0.8in]{geometry}
\usepackage{setspace}
\usepackage[utf8]{inputenc}
\usepackage[colorlinks=true,urlcolor=Blue,citecolor=Blue,linkcolor=BrickRed]{hyperref}
\usepackage[dvipsnames,usenames]{xcolor}
%\spacing{1.06}

\newcommand{\handout}[5]{
  \noindent
  \begin{center}
  \framebox{
    \vbox{\vspace{0.25cm}
      \hbox to 5.78in { {University of Wrocław:\hspace{0.12cm}Algorithms for
          Big Data (Fall'19)} \hfill #2 }
      \vspace{0.48cm}
      \hbox to 5.78in { {\Large \hfill #5  \hfill} }
      \vspace{0.42cm}
      \hbox to 5.78in { {#3 \hfill #4} }\vspace{0.25cm}
    }
  }
  \end{center}
  \vspace*{4mm}
}

\newcommand{\lecture}[4]{\handout{#1}{#2}{#3}{Scribe:\hspace{0.08cm}#4}{Lecture #1}}

\newtheorem{theorem}{Theorem}
\newtheorem{corollary}[theorem]{Corollary}
\newtheorem{lemma}[theorem]{Lemma}
\newtheorem{observation}[theorem]{Observation}
\newtheorem{example}[theorem]{Example}
\newtheorem{definition}[theorem]{Definition}
\newtheorem{claim}[theorem]{Claim}
\newtheorem{fact}[theorem]{Fact}
\newtheorem{assumption}[theorem]{Assumption}

\newcommand{\E}{{\mathbb E}}
\DeclareMathOperator{\var}{Var}
\newcommand{\eps}{\varepsilon}
\newcommand{\bigo}{\mathcal{O}}
\setcounter{MaxMatrixCols}{20}

\begin{document}

\lecture{10: Compressed Sensing}{ 16/12/2019}{Lecturer: \emph{Przemysław Uznański }}{-}


\section{Overview}
A version of sparse recovery: given linear measurements, recover the signal. In general impossible if $m<n$, but is possible if we make additional assumptions.

\begin{definition}
Given $A \in \mathbb{R}^{m \times n}$, and $y = Ax \in \mathbb{R}^m$, recover $x \in \mathbb{R}^n$ given that $x$ is $k$-sparse.
\end{definition}

Rephrasing problem:
$$\text{minimize}\quad\|x'\|_0$$
$$\text{subject to}\quad Ax' = Ax$$
(this is hard problem, NP-hard?)

The approach: solve the following program
$$\text{minimize}\quad\|x'\|_1$$
$$\text{subject to}\quad Ax' = Ax$$

The insight is that signals with small  $\|x'\|_1$ should be sparse (the weight of $x'$ is concentrated on a few ``optimal'' coordinates). Particularly, optimizing $\|x'\|_2$ would not work.

(Adding noise: $\text{subject to}\quad \|Ax - Ax'\|_2 \le \varepsilon$.)

The program can be expressed as LP, with $n^{\Theta(1)}$ algorithm.


\begin{theorem}
Let $m = \Theta(k \log(n/k))$. Pick $A \in \mathbb{R}^{m \times n}$ with entries $\mathcal{N}(0,1)$. W.h.p. the following holds for some $C$: the output $x'$ of the LP satisfies
\begin{equation}
\label{eq:ref}
\|x'-x\|_2 \le \frac{C}{\sqrt{k}} \min_{\|x''\|_0 \le k} \|x - x''\|_1.
\end{equation}

In fact any JL-type distribution works.
\end{theorem}

\subsection{Restricted Isometry Property}

\begin{definition}
We say that matrix $A$ satisfies $(k,\varepsilon)$-RIP if for any $k$-sparse $x$ there is $$\|Ax \|_2 = (1\pm \varepsilon) \|x\|_2.$$
\end{definition}

\begin{theorem}
JL matrix with $m = \Theta(k \log(n/k))$ satisfies $(k,1/3)$-RIP.
\end{theorem}
\begin{proof}
Recall proof of subspace embeddings. Here as well it is enough to show the property for $\|x\|_2 = 1$. For any $K \subseteq [n]$, $|K| = k$, we take $B_K = \{ x : \|x\|_2 = 1 \wedge \textrm{support}(x) \subseteq K \}$, that is vectors spanned on coordinates in $K$ of length $1$.

Pick $(1/7)$-net $N_K$ for $B_K$ of size $2^{\bigo(k)}$. For any $x \in B_K$ there is
$x = \sum_{i\ge0} \alpha_i x_i $ where $\alpha_i \le 7^{-i}$ and $x_i \in B_K$, and $\alpha_0 = 1$.

For $A$ properly set (JL with constant $1/7$), there is
$$\|Ax\|_2 = \|A \sum_{i\ge0} \alpha_i x_i\|_2 \le \sum_i \alpha_i \|A x_i\|_2 \le \sum_i \frac{8}{7} \cdot 7^{-i} \le \frac{4}{3}$$
and
$$\|Ax\|_2 = \|A \sum_{i\ge0} \alpha_i x_i\|_2 \ge \|A x_0\|_2 - \sum_{i\ge1} \alpha_i \|A x_i\|_2  \ge \frac{6}{7} - \sum_{i\ge1} \frac{8}{7} 7^{-i} = \frac{6}{7} - \frac{4}{3} \cdot \frac{1}{7} = \frac23$$

How large the $m$ needs to be? It needs to work for union of all $1/7$-nets. We have $|\bigcup N_K| \le {n \choose k} \cdot 2^{\bigo(k)} = 2^{\bigo(k \log(n/k))}$.
\end{proof}

\subsection{Basis pursuit}
\begin{theorem}
$(4k,1/3)$-RIP implies \eqref{eq:ref}.
\end{theorem}
\begin{proof}
Denote $h = x'-x$. Let us reorder $x$ so that $x_1,\ldots,x_k$ are the largest magnitude coefficients, and then so that $h_{k+1},\ldots$ are sorted (ignoring that $h_1,\ldots,h_k$ might be arbitrary). Denote $A = \{1,\ldots,k\}$ and $A_j = \{k+5jk+1,\ldots,6k+5jk\}$ for $j \ge 0$.

Denote $\eta = \min_{\|x''\|_0 \le k} \|x - x''\|_1 = \|x_{A^c}\|_1$.

First, bound:
$$\|x\|_1 \ge \|x'\|_1 = \|x + h\|_1 = \|x_A+h_A\|_1 + \|x_{A^c} + h_{A^c}\|_1 \ge \|x_A\|_1 - \|h_A\|_1 - \|x_{A^c}\|_1 + \|h_{A^c}\|_1$$
so
\begin{equation}
\label{eq:2}
\|h_{A^c}\|_1 \le \|h_{A}\|_1 + 2 \|x_{A^c}\| = \|h_{A}\|_1 + 2 \eta
\end{equation}

Then for any vector $y$ with sorted coordinates, $|y_i| \le \|y\|_1/i$. So
$$
\|h_{(AA_0)^c}\|_2^2 \le \|h_{A^c}\|_1^2 \sum_{j=5k}^{\infty} \frac{1}{j^2} \le  \|h_{A^c}\|_1^2 \frac{1}{\bigo(k)}
$$
so, using \eqref{eq:2}
\begin{equation}
\|h_{(AA_0)^c}\|_2 = \bigo(\frac{\|h_{A}\|_1 + \eta}{\sqrt{k}})
\end{equation}

Then
$$0 = \|A(x'-x)\|_2 = \|Ah\|_2 \ge \|A h_{AA_0}\|_2 - \|\sum_{j\ge1} A h_{A_j} \|_2$$
$$ \ge  \|A h_{AA_0}\|_2 - \sum_{j\ge1} \|A h_{A_j} \|_2$$
$$\ge \frac23  \|h_{AA_0}\|_2 - \frac43\sum_{j\ge1} \| h_{A_j} \|_2$$

Each term in $h_{A_{j+1}}$ is smaller than average term in $h_{A_j}$, so $\|h_{A_{j+1}}\|_2^2 \le (5k) \frac{\|h_{A_j}\|_1^2}{(5k)^2} =  \frac{\|h_{A_j}\|_1^2}{5k}$. So
\begin{equation}
\label{eq:4}
\sum_{j\ge1} \| h_{A_j} \|_2 \le \sum_{j \ge 0} \frac{\|h_{A_j}\|_1}{\sqrt{5k}} = \frac{\|h_{A^c}\|_1}{\sqrt{5k}} =  \frac{\|h_{A}\|_1 + 2\eta}{\sqrt{5k}} \le \sqrt{\frac{k}{5k}} \|h_{A}\|_2 + \frac{2}{\sqrt{5k}} \eta
\end{equation}
Combining we have
$$  \frac{2}{3} \|h_{AA_0}\|_2 \le \frac{4}{3 \sqrt{5}} \|h_A\|_2 + \bigo(\frac{\eta}{\sqrt{k}})$$
so since $\|h_A\|_2 \le \|h_{AA_0}\|_2$ and $\frac{2}{3} < \frac{4}{3 \sqrt{5}}$, there is
$$ \|h_{AA_0}\|_2 = \bigo(\frac{\eta}{\sqrt{k}})$$
and this bound also applies in the form of
$$\frac{\|h_{A}\|_1 }{\sqrt{k}} = \bigo(\frac{\eta}{\sqrt{k}}).$$
By triangle inequality
$$\|h\|_2 \le \|h_{AA_0}\|_2 + \|h_{(AA_0)^c}\|_2 =  \bigo(\frac{\eta}{\sqrt{k}})$$
\end{proof}
\subsection{Iterative Hard Thresholding \cite{4960344}}

Assume $y = Ax+e$, where $e$ is some noise, and $x$ is $k$-sparse.

The algorithm works as follow. 
\begin{itemize}
\item $x^0 = 0$.
\item $a^{i+1} = x^i + A^T(y - Ax^i)$
\item $x^{i+1} = H_k[a^{i+1}]$ where $H_k$ is projection onto top-$k$ coefficients.
\end{itemize}

We expect $x^t$ to converge onto $x \pm \bigo(e)$, in logarithmic number of iterations.
Proof:

denote $r^i = x - x^i$.  Our goal is to show that $r^i$ decreases by a constant factor at each step.

Denote $S = \text{support}(x)$ and $S^i = \text{support}(x^i)$, and $B^i = S \cup S^i$.

We assume we have $(3k,\varepsilon)$-RIP for sufficiently small $\varepsilon$ (like, small constant).

Properties: ($A_S$ denotes matrix with only columns in $S$ kept, and all others zeroed out).
\begin{lemma}
For $S$ such that $|S| \le 3k$:
$$\|(I - A_S^T A_S)x_S\|_2 \le \varepsilon \|x_S\|_2$$
\end{lemma}
\begin{proof}
Follows from $A_S$ being subspace embedding for $3k$-sparse vectors.
\end{proof}

\begin{lemma}
for any disjoint $S,S'$ with $|S|+|S'| \le 3k$
$$\|A_S^TA_{S'}x_{S'}\|_2 \le \varepsilon \|x_{S'}\|_2$$
\end{lemma}
\begin{proof}
$$\|A_S^T A_{S'}\|_2 = \sup_{\|y\|=1} \|A_S^T A_{S'} y\|_2 = \sup_{\|x\|=1,\|y\|=1} x^T A_S^T A_{S'} y =  \sup_{\|x\|=1,\|y\|=1} \langle A x_S , A y_{S'} \rangle $$
$$=  \sup_{\|x\|=1,\|y\|=1} ( \langle x_S ,  y_{S'} \rangle \pm \varepsilon ) \le \varepsilon$$

where we used that $\langle x_S, y_{S'} \rangle = 0$ and that linear combinations of $x_S,y_{S'}$ are also sparse.
\end{proof}

\begin{lemma}
$$\|A^T_S y\|_2^2 \le (1+\varepsilon)\|y\|_2^2$$
\end{lemma}
\begin{proof}
Follows from $\|A_S^T\|_2 = \|A_S\|_2 \le (1+\varepsilon)$, which follows from:
$$\|A_S\|_2 = \sum_{\|x\|_2 = 1} \|A_S x_S\|_2 \le \sqrt{1+\varepsilon} \|x_S\| = \sqrt{1+\varepsilon}$$
\end{proof}


Now we bound $\|r^{i+1}\|_2$:

$$\|r^{i+1}\|_2 = \|x-x^{i+1}\|_2 = \|x_{B^{i+1}} - x^{i+1}_{B^{i+1}}\|_2 \le  \|x_{B^{i+1}} - a^{i+1}_{B^{i+1}}\|_2 + \|x^{i+1}_{B^{i+1}} - a^{i+1}_{B^{i+1}}\|_2$$
since $x^{i+1}_{B^{i+1}}$ is the best possible $k$-sparse approximation to $a^{i+1}_{B^{i+1}}$, and $x_{B^{i+1}}$ is $k$-sparse, we have
$$\|r^{i+1}\|_2 \le 2 \|x_{B^{i+1}} - a^{i+1}_{B^{i+1}}\|_2 $$
$$\le 2\|x_{B^{i+1}} - x^i_{B^{i+1}} - A^T_{B^{i+1}} A r^i  - A^T_{B^{i+1}} e\|_2$$
$$\le 2\|r^i_{B^{i+1}} - A^T_{B^{i+1}} A r^i \|_2 + 2 \|A^T_{B^{i+1}} e\|_2$$
$$\le 2\|(I - A^T_{B^{i+1}}A_{B^{i+1}})r^i_{B^{i+1}}\|_2  + 2 \|A^T_{B^{i+1}} A_{B^i-B^{i+1}} r^i_{B^i-B^{i+1}} \|_2+ 2 \sqrt{1+\varepsilon}\| e\|_2$$
$$\le 2\varepsilon\|r^i_{B^{i+1}}\|_2  + 2 \varepsilon\| r^i_{B^i-B^{i+1}} \|_2 + 2 \sqrt{1+\varepsilon}\| e\|_2$$
$$\le 4\varepsilon\|r^i\|_2 + 2 \sqrt{1+\varepsilon}\| e\|_2$$
set $\varepsilon = 1/8$
$$\le \frac12\|r^i\|_2 + 3\|e\|_2$$

\bibliographystyle{alpha}
\bibliography{bib}

\end{document}

