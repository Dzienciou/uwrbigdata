\documentclass[12pt]{uebung}


\dozent{Przemysław Uznański}
\vorlesung{Algorithms for Big Data}
\semester{Fall Semester 2019}
%\tutoren{Tutoren name}
 
\usepackage{amsmath}
\usepackage{mathtools}
 
 
%\usepackage[utf8]{inputenc}
\usepackage{amsmath, amssymb,wasysym}
%\usepackage{tikz}
\newtheorem{definition}{Definition}
\newtheorem{theorem}{Theorem}

 
%\usepackage{multirow}

\usepackage[english]{babel}
 \begin{document}

 \startnummer{1}
 
\kopf[0]{28/10/2019}{4}


\renewcommand{\aufgname}{Exercise}
\begin{aufg}
Show that Cauchy distribution $f(x) = \frac{1}{\pi(1+x^2)}$ is 1-stable, that is for $X,Y \sim \textsf{Cauchy}$ and $a,b \in \mathbb{R}$ we have $a*X+b*Y \sim (|a|+|b|) \cdot \textsf{Cauchy}$.
\end{aufg}

\begin{aufg}
Modify the algorithm for computing $F_p$ of a stream (the one working for any $p$ in the sequence of values) to work in semi-turnstile streams (updates $(x_i,c_i)$ where $c_i \in \mathbb{R}^+$).
\end{aufg}

\bigskip
\bigskip

The goal of next few exercises is to show that $p$-stable distribution approach can be used for sketching of Hamming norm/distance. We follow the analysis from \emph{''Comparing Data Streams Using Hamming Norms
(How to Zero In)''} by Cormode, Datar, Indyk and Muthukrishnan (VLDB'02).\footnote{With the only change that we actualy want to do the correct analysis - the analysis in the paper has an error.}

\begin{aufg}
Assume we operate in universe $U$ of magnitude $u$, that is we have a promise that all our values we are ever going to see when sketching are integers from $\{-u,\ldots,1,0,1,\ldots,u\}$. Show that for sufficiently small value $p$, $F_p$ is an $1 \pm \varepsilon$ approximation of $F_0$. Show that $p= \Theta(\varepsilon/\log u)$ is small enough.

Show that $u$-factor approximation of $L_p$ approximation is enough to obtain $1 \pm \varepsilon$ approximation of $F_p$, and thus $1 \pm 2\varepsilon$ of $F_0$.
\end{aufg}

\bigskip
\bigskip


Useful fact: when $p \to 0$, we have $\Pr( |X| > t) = \Theta(t^{-p})$ for $X$ drawn from $p$-stable, 0 mean, normalized by median stable distribution.

\begin{aufg}[2 pts]
Take value of $p$ from Exercise 3 and desired level of approximation. How many samples do we need to take from this $p$-stable distribution to reach this level of approximation for median estimation?
What is the memory complexity of our algorithm? Take into account the actual bit-size of bignums needed in our algorithm.\footnote{And here is the omission of the authors...}
\end{aufg}
\end{document}
