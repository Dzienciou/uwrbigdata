\documentclass[12pt]{uebung}


\dozent{Przemysław Uznański}
\vorlesung{Algorithms for Big Data}
\semester{Fall Semester 2019}
%\tutoren{Tutoren name}
 
\usepackage{amsmath}
\usepackage{mathtools}
 
 
%\usepackage[utf8]{inputenc}
\usepackage{amsmath, amssymb,wasysym}
%\usepackage{tikz}
\newtheorem{definition}{Definition}
\newtheorem{theorem}{Theorem}

 
%\usepackage{multirow}

\usepackage[english]{babel}

 \begin{document}

 \startnummer{1}
 
\kopf[0]{16/12/2019}{10}

\newcommand{\bigo}{\mathcal{O}}
\renewcommand{\aufgname}{Exercise}

Below we assume that $k \ll m \ll n$ to avoid annoying border-cases.
%TODO $\le (k+1)$-separable implies $k$-disjointness

\begin{aufg}
Show that any $k$-disjoint set family separates $I_1,I_2$ such that $I_1 \not= I_2$ and $I_2$ can be arbitrarily large, while $|I_1| \le k$.
\end{aufg}

\begin{aufg}
Describe a decoding procedure for $k$-disjoint set family: given $\bigcup_{i \in I} F_i$, output $I$ if $|I| \le k$, and otherwise outputs that its not the case.
\end{aufg}

\begin{aufg}
Let $A$ be a $k$-disjoint matrix.  Show a decoding procedure, that given $Ax$ outputs $x$ if $x$ is $k$-sparse, and otherwise outputs that its not the case. Assume $x \ge 0$.
\end{aufg}

\begin{aufg}[2 pts]
Assume $k$-disjoint family which has slow decoding. Show that it can be transformed into (suboptimal) $k$-separable family with $m' = \bigo(m \log n)$, and decoding time $\textrm{poly}(m,k,\log n)$. (You might want to expand the universe over which we define our family.)
\end{aufg}
\end{document}
