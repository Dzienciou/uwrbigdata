\documentclass[11pt]{article}
\usepackage{latexsym}
\usepackage{amsmath,amssymb,amsthm}
\usepackage{epsfig}
\usepackage[right=0.8in, top=1in, bottom=1.2in, left=0.8in]{geometry}
\usepackage{setspace}
\usepackage[utf8]{inputenc}
\usepackage{url}
%\spacing{1.06}

\newcommand{\handout}[5]{
  \noindent
  \begin{center}
  \framebox{
    \vbox{\vspace{0.25cm}
      \hbox to 5.78in { {University of Wrocław:\hspace{0.12cm}Algorithms for
          Big Data (Fall'19)} \hfill #2 }
      \vspace{0.48cm}
      \hbox to 5.78in { {\Large \hfill #5  \hfill} }
      \vspace{0.42cm}
      \hbox to 5.78in { {#3 \hfill #4} }\vspace{0.25cm}
    }
  }
  \end{center}
  \vspace*{4mm}
}

\newcommand{\lecture}[4]{\handout{#1}{#2}{#3}{Scribes:\hspace{0.08cm}#4}{Lecture #1}}

\newtheorem{theorem}{Theorem}
\newtheorem{corollary}[theorem]{Corollary}
\newtheorem{lemma}[theorem]{Lemma}
\newtheorem{observation}[theorem]{Observation}
\newtheorem{example}[theorem]{Example}
\newtheorem{definition}[theorem]{Definition}
\newtheorem{claim}[theorem]{Claim}
\newtheorem{fact}[theorem]{Fact}
\newtheorem{assumption}[theorem]{Assumption}
\newtheorem{remark}[theorem]{Remark}

\newcommand{\E}{{\mathbb E}}
\DeclareMathOperator{\var}{Var}

\begin{document}

\lecture{3: Sketches for $L_p$ norms}{21/10/2019}{Lecturer: \emph{Przemysław Uznański }}{\emph{Wiktor Garbarek}}

\section{$p$-stable distributions}
\begin{definition}
  A distribution $\mathcal{D}$, with mean $0$, is called stable, if for $X_1,..,X_n \sim \mathcal{D}$ which are independent and $a_1,..,a_n \in \mathbb{R}$ there is $\sum_i a_i X_i = b \cdot Z$ for some $b \in \mathbb{R}$ and $Z \sim \mathcal{D}$.
\end{definition}
\begin{definition}
  A distribution D is $p$-stable if is stable and coefficient $b$ from previous definition satisfies\[ b = \Big(\sum_i |x_i|^p\Big)^{1/p} \]
\end{definition}

\begin{remark}
(Zolotarev, 1986) $p$-stable distribution exists if and only if $0 < p \leq 2$.

For $p \in \{\frac{1}{2},1,2\}$ we know closed form formulas, e.g.:
\begin{itemize}
  \item Normal distribution, that is $f(x) = \frac{e^{-x^2/2}}{\sqrt{2 \pi}}$, is 2-stable.
  \item Cauchy distribution, that is $f(x) = \frac{1}{(1+x^2) \pi}$, is 1-stable.
  \item Lévy distribution is $\frac{1}{2}$-stable.
\end{itemize}
\end{remark}

\begin{remark}
  Except for $p=2$, those distributions are heavy tailed, that is $\E[|\mathcal{D}|] = \infty$ and $\E[\mathcal{D}^2] = \infty$ (this has to be, as by Central Limit Theorem distributions with finite moments cannot be stable, unless its normal distribution).
\end{remark}

\section{Sketches for $L_p$ norm [Indyk 2000]}

Pick random coefficients $r_i \sim \mathcal{D}_p$ for  $i = 1 .. n$, where $\mathcal{D}_p$ is a $p$-stable distribution. Then
$$Z = \sum_i x_i r_i$$ is a sketch of vector $\mathbf{x}$, since $Z \sim |\mathbf{x}|_p \mathcal{D}_p$. We will of course run many parallel instances of sketching process (as usual).

\begin{remark}
  Our sketches are linear functions, so linear combination of sketches is also a sketch.
\end{remark}


\medskip
\noindent Whenever update $(x_i, c_i)$ comes, to maintain the sketch we compute $Z := Z + c_i \cdot r_{x_i}$ 

Challenges:
\begin{itemize}
 \item How to draw random values from $p$-stable distribution?
 \item How to extract the result?
 \item How much independence is required?
\end{itemize}
\subsection{Drawing from $p$-stable}
\begin{itemize}
\item $p=1$:  If $U \sim \mathcal{U}(0,1)$ then $\tan{\pi(U - \frac{1}{2})}$ is distributed with the Cauchy distribution
\item $p = 2$:
  We can use Box-Muller transformation. If $U, V \sim \mathcal{U}(0,1)$ and are iid, then $\sqrt{-2\ln{U}}\cdot\cos{2\pi V}$ is distributed as a normal distribution.
\item $p \in (0,2)$ and $p \neq 1$:
  In this case p-stable distribution can be simulated by method derived by Chambers, Mallows and Stuck (1976). 
  If $U, V \sim \mathcal{U}(0,1)$ and are iid and let's set $\Theta(U) = \pi (U - \frac{1}{2})$. Then 
  $$
  \frac{\sin{p\Theta(U)}}{\cos^{\frac{1}{p}}{\Theta(U)}} \Bigg(\frac{\cos{(\Theta(U)\cdot (1 - p))}}{- \ln V}\Bigg)^{\frac{1-p}{p}}
  $$
  is distributed as a p-stable distribution.
\end{itemize}

\subsection{Extracting the result via median}
Recall $Z \sim |\mathbf{x}|_p \mathcal{D}_p$
Expected value is useless, since it is infinite, so let's consider median: $$\textrm{median}(|Z|) \sim |\mathbf{x}|_p \cdot \textrm{median}(|\mathcal{D}_p|)$$ 

\medskip
\noindent How to extract median of a distribution $X$ (on $\mathbb{R}_+$)? 

Let $F$ be CDF of distribution $X$. Let's also sample $k$ values $x_1,\dots,x_k \sim X$, and output $\textrm{median}(x_1,\dots,x_k)$. 
By Chernoff bound, if we have
$k = O(\log(\frac{1}{\delta})/\varepsilon^2)$, then $$F^{-1}(\frac{1}{2}-\varepsilon) \leq \textrm{median}(x_1,\dots,x_k) \leq F^{-1}(\frac{1}{2}+\varepsilon)$$
\medskip
If $F^\prime$ is not too flat around $F^{-1}(\frac{1}{2})$ (i.e. median), then $F^{-1}(\frac{1}{2} \pm \varepsilon)$ are actually $1\pm C\cdot\varepsilon$ approximations of median (required $F^\prime(x) \geq \frac{1}{C}$ in a given range). This becomes an issue when $p \rightarrow 0$, but we don't have to care for constant $p$.

\medskip
\noindent
Estimator: $\frac{\textrm{median}(|Z|)}{\textrm{median}(|D_p|)} = \frac{\textrm{median}(|Z_1|,\ldots,|Z_k|)}{\textrm{median}(|D_p|)}$


\subsection{Geometric mean estimator [Li 2008]}
Use geometric mean as an estimator. 

Output: $(\Pi_{i=1}^k|Z_i|)^{1/k} / \alpha$, where
$\alpha = e^{\E(\ln |D_p| )}$, and $k = O(\log(\frac{1}{\delta}) / \varepsilon^2)$.

\subsection{Independence}
To use $p$-stability we assumed full $n$-wise independence - this means in theory huge space in streaming applications, and there is no easy workaround.  We present two workarounds that are theoretic in nature and require very sophisticated machinery.
\subsubsection{First workaround [Indyk 2000]}
Apply Nisan’s pRNG which is able to:
\begin{itemize}
\item Store its seed/local state in a $\textrm{polylog}(n)$ space
\item Extract its next random number in a small working space and small time
\item “Fool” small space (arbitrary) computations as if they were given perfect randomness
\end{itemize}

\subsubsection{Second workaround [Kane, Nelson and Woodruff 2008]}
Prove that $k$-wise independence is enough, for $$k = \mathcal{O}( \log \frac{1}{\varepsilon} \log\log\log \frac{1}{\varepsilon})$$

\section{$L_p$ sketching for $p > 2$}
\subsection{2-pass Algorithm}
Algorithm (only for sequence of values, not for turnstile):
  \begin{enumerate}
    \item Pick uniformly and at random $i \in [n]$
    \item In the first pass: Select $x = x_i$
    \item In the second pass: Compute $f_{x} = |\{ j : x_j = x \}|$
    \item return output $Y = n (f_x)^{p-1}$
  \end{enumerate}
$$E[Y] = \frac{1}{n}\sum_{i} n(f_{x_i})^{p-1} = \sum_{x} (f_x)^{p-1} f_x = \sum_{x} (f_x)^p = F_p$$
$$E[Y^2] = \frac{1}{n}\sum_{i}n^2 (f_{x_i})^{2p-2} = n \sum_{x} (f_x)^{2p-2} f_x = n F_{2p-1}$$

\begin{claim}
  Following inequality holds $$n F_{2p-1} \leq m^{1-1/p} (F_p)^2$$
\end{claim}
\begin{proof}
$$nF_{2p-1} = n \|f\|_{2p-1}^{2p-1} \leq n \|f\|_{p}^{2p-1} = \|f\|_1 \|f\|_p^{2p-1} \leq m^{1 - 1/p} \|f\|_p \|f\|_p^{2p-1} = m^{1 - 1/p} \|f\|_p^{2p} = m^{1 - 1/p} F_p^2$$
For $\|f\|_1 \leq m^{1 - 1/p} \|f\|_p$ and $\|f\|_{2p-1} \le \|f\|_p$ see Wikipedia\footnote{\url{https://en.wikipedia.org/wiki/Lp_space#Relations_between_p-norms}}.
\end{proof}
\subsection{1-pass Algorithm}
How to pick random $i$ in a stream of unknown length?
\begin{enumerate}
  \item Initialize $r \gets 0$.
  \item At $i$-th step of algorithm: with probability $\frac{1}{i}$ we set $x \gets x_i$ and $r \gets 0$
  \item If $x_{i} = x$ then $r \gets r + 1$
  \item Return output $Y^\prime = n(r^p - r^{p-1})$
\end{enumerate}
$$E[Y^\prime] = \sum_x \sum_{r \leq f_x} (r^p-r^{p-1}) = \sum_x (f_x)^p = F_p
Y^\prime \leq n p r^{p-1} \leq n p (f_x)^{p-1} = p Y$$
$$E[(Y^\prime)^2] \leq p^2 E[Y^2] = p^2 m^{1-1/p} (F_p)^2$$
And thus the number of parallel runs should be $\mathcal{O}\Big(\frac{p^2 m^{1 - \frac{1}{p}}}{\varepsilon^2}\Big)$

\begin{fact}
  There is a lowerbound for size of sketches for $L_p$, $p > 2$: $\Omega(m^{1-\frac{2}{p}}/\varepsilon^2)$ [see Bar-Yossef, Jayram, Kumar, and Sivakumar 2004]
\end{fact}
\begin{fact}
  Best complexity of sketching is $\mathcal{O}(m^{1-\frac{2}{p}}/\varepsilon^2)$ -- achieved in turnstile model [Indyk and Woodruff 2005].
\end{fact}
\end{document}

