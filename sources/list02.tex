\documentclass[12pt]{uebung}


\dozent{Przemysław Uznański}
\vorlesung{Algorithms for Big Data}
\semester{Fall Semester 2019}
%\tutoren{Tutoren name}
 
\usepackage{amsmath}
\usepackage{mathtools}
 
 
%\usepackage[utf8]{inputenc}
\usepackage{amsmath, amssymb,wasysym}
%\usepackage{tikz}
\newtheorem{definition}{Definition}
\newtheorem{theorem}{Theorem}

 
%\usepackage{multirow}

\usepackage[english]{babel}
 \begin{document}

 \startnummer{1}
 
\kopf[0]{14/10/2019}{2}


\renewcommand{\aufgname}{Exercise}

Pseudorandomness: emulating perfect randomness in a predictable manner. Recall a following measure of \emph{quality}
\begin{definition}
Consider a family of hash functions $\mathcal{H} = \{h : [u] \to [m]\}$.\footnote{$[u] = \{0,1,\ldots,u-1\}$ is called an \emph{universe}.} We say that $\mathcal{H}$ is $k$-wise independent if for any distinct $x_1,\ldots,x_k \in [u]$ and any (not necessarily distinct) $y_1,y_2,\ldots,y_k \in [m]$ there is
$$\Pr_{h\in H}( h(x_1)=y_1 \wedge \ldots \wedge h(x_k) = y_k ) = \Theta(m^{-k}).$$
\end{definition}
Informally: those hash-functions are indistinguishable from perfectly random hashing when evaluated simultaneously at $k$ values. 

We claim that (i) $k$-wise independence is good enough to ''fool'' algorithms into behaving as if provided with perfect randomness and (ii) this type of pseudo-randomness can be stored using small space.


\begin{aufg}
Let $p > u$ be prime number. 
Let $\mathbf{a} = a_0,..,a_{k-1}$ be vector of coefficients. Let $h_{\mathbf{a}} : [u] \to [m]$ be defined as $h_{\mathbf{a}}(x)=[(\sum_{i=0}^{k-1}a_ix^i) \text{ mod } p ] \text{ mod } m$. Show that $\mathcal{H} = \{h_{\mathbf{a}}\ |\ a_0,..,a_{k-1} \in [p]\}$ is k-wise independent.

\textbf{\em Hint}: \\
Polynomial of degree $k-1$ in $\mathbb{Z}_p$ is uniquely defined by its value on $k$ distinct points.
\end{aufg}

\begin{aufg}
Show that families of hash-functions from previous exercise are not $(k+1)$-wise independent.
\end{aufg}

\begin{aufg}
Show a lower-bound of $\Omega(k \log m)$ bits necessary to represent (store) a hash-function from $k$-wise independent hash-function family. How much space do we need to represent perfectly random hash-function? 
\end{aufg}

\begin{aufg}
Let $X_1,X_2,\ldots,X_n$ be pairwise independent random variables. Show that $\mathrm{Var}[ \sum_i X_i ] = \sum_i \mathrm{Var}[ X_i ]$.
\end{aufg}

\newpage

\begin{aufg}
Missing part of Morris' algorithm analysis: show inductively that $\mathbb{E}\left[\left(2^{X_n}\right)^2\right] = \frac{3}{2}n^2 + \frac{3}{2} n + 1$.
\end{aufg}

\begin{aufg}[2 pts]
Consider following idea for concentrating Flajolet-Martin approach. Let $r_1,r_2,\ldots,r_n \in [0,1]$ be picked uniformly and independently at random, and let $X_k$ be $k$-th smallest value among $r_1,\ldots,r_n$. Find $\mathbb{E}[X_k]$ and $\mathrm{Var}[X_k]$. Use it to derive streaming algorithm for distinct elements (see Bar-Yossef et al. 2002).
\end{aufg}
\end{document}
