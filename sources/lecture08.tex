\documentclass[11pt]{article}
\usepackage{latexsym}
\usepackage{amsmath,amssymb,amsthm}
\usepackage{epsfig}
\usepackage[right=0.8in, top=1in, bottom=1.2in, left=0.8in]{geometry}
\usepackage{setspace}
\usepackage[utf8]{inputenc}
%\spacing{1.06}

\newcommand{\handout}[5]{
  \noindent
  \begin{center}
  \framebox{
    \vbox{\vspace{0.25cm}
      \hbox to 5.78in { {University of Wrocław:\hspace{0.12cm}Algorithms for
          Big Data (Fall'19)} \hfill #2 }
      \vspace{0.48cm}
      \hbox to 5.78in { {\Large \hfill #5  \hfill} }
      \vspace{0.42cm}
      \hbox to 5.78in { {#3 \hfill #4} }\vspace{0.25cm}
    }
  }
  \end{center}
  \vspace*{4mm}
}

\newcommand{\lecture}[4]{\handout{#1}{#2}{#3}{Scribe:\hspace{0.08cm}#4}{Lecture #1}}

\newtheorem{theorem}{Theorem}
\newtheorem{corollary}[theorem]{Corollary}
\newtheorem{lemma}[theorem]{Lemma}
\newtheorem{observation}[theorem]{Observation}
\newtheorem{example}[theorem]{Example}
\newtheorem{definition}[theorem]{Definition}
\newtheorem{claim}[theorem]{Claim}
\newtheorem{fact}[theorem]{Fact}
\newtheorem{assumption}[theorem]{Assumption}

\newcommand{\E}{{\mathbb E}}
\DeclareMathOperator{\var}{Var}
\newcommand{\eps}{\varepsilon}
\newcommand{\bigo}{\mathcal{O}}
\setcounter{MaxMatrixCols}{20}

\begin{document}

\lecture{8: Sparse FFT}{ 02/12/2019}{Lecturer: \emph{Przemysław Uznański }}{-}


\section{Sparse FFT - no noise}
Assume wlog $n$ is power of two.

\paragraph{Plan of solution:}

Define $p_{d,\ell}(x) = \sum_{i : i \bmod 2^{\ell} = d} \hat{a}_i x^i$, and for short $p(x) = p_{0,0}(x) = \sum_i \hat{a}_i x^i$. 

 and for a polynomial $f(x)$ we write for any polynomial $\|f\|_2^2 = \sum_i f_i^2$. 

\begin{enumerate}
\item Define $S_{\ell} = \{i : \|p_{i,\ell}\|^2 > 0 \}$
\item Given $S_{\ell}$, compute $S_{\ell+1}$: for each $d \in S_{\ell}$, test for $e \in \{d, d+2^{\ell}\}$ whether $\|p_{e,\ell+1}\|^2 > 0$ and if so, add $e$ to $S_{\ell+1}$.
\item $p_{i,\log n}(1) = \hat{a}_i$
\end{enumerate}

The idea is to start at level 0 and proceed. The size of each $S_{\ell}$ is bounded by $k$, thus the total number of steps (2) done is $\bigo(k \log n)$.


\paragraph{A few identities:}
\ \\

By inverse Fourier transform
$$p(\omega^t)/\sqrt{n} = \frac{1}{\sqrt{n}}\sum_i \hat{a}_i \omega^{it} = a_{-t}$$

By Parseval's theorem
$$\|p\|_2^2 = \sum_i \hat{a}_i^2 = \sum_i a_i^2 = \frac{1}{n}\sum_i |p(\omega^i)|^2$$
so
$$\|p_{d,\ell}\|_2^2 = \frac{1}{n}\sum_{i=0}^{n-1} |p_{d,\ell}(\omega^i)|^2$$

Additionally
\begin{align*}
\frac{1}{2^\ell} \sum_{i=0}^{2^\ell-1} p(\omega^{t+i\frac{n}{2^\ell}}) \omega^{-d i \frac{n}{2^\ell}} 
&= \frac{1}{2^\ell} \sum_{i=0}^{2^\ell-1} \sum_{j=0}^{n-1}  \hat{a}_j (\omega^{t+i\frac{n}{2^\ell}})^j \omega^{-d i \frac{n}{2^\ell}} \\
&= \sum_{j=0}^{n-1}  \hat{a}_j \omega^{tj}\frac{1}{2^\ell} \sum_{i=0}^{2^\ell-1}  \omega^{i(j-d)\frac{n}{2^\ell}} \\
&=\sum_{j=0}^{n-1} \hat{a}_j \omega^{tj} [j-d = 0 \bmod 2^\ell]\\
&= p_{d,\ell}(\omega^t)
\end{align*}


\paragraph{Estimating sum via sampling:}
Lets say we have $a_1, \ldots, a_n$ such that $|a_i| \le H$.  We can use $\bigo(H^2/\varepsilon^2 \cdot \log n)$ samples to obtain $\pm \varepsilon$ estimate of  $\frac{1}{n}\sum_i a_i$, or equivalently $\pm n \varepsilon$ estimate of $\sum_i a_i$, with proof via Hoeffding bound.


\paragraph{First solution:}
Denote $L = \max |\hat{a}_i|$ and $H = \max |p(\omega^t)|$. Observe that $p_{d,\ell}(\omega^t) \le H$. Since $p(\omega^t) = \sum_j \hat{a}_j \omega^{-tj}$, we have $H \le kL$.
\begin{enumerate}
\item Estimate $A_{d,\ell,t} \approx p_{d,\ell}(\omega^t)$ using $\bigo(H^4 \log n)$ samples of $a_{-t - i\frac{n}{2^\ell}} \omega^{-d i \frac{n}{2^\ell}}$, up to error $\pm \frac{1}{16H}$.
\item $|A_{d,\ell,t}|^2 - |p_{d,\ell}(\omega^t)|^2 \le  (|A_{d,\ell,t}| - |p_{d,\ell}(\omega^t)|)(|A_{d,\ell,t}| + |p_{d,\ell}(\omega^t)|) \le \frac{1}{16H} \cdot 2H \le \frac{1}{4}$
\item $\|p_{d,\ell}\|_2^2 = \frac{1}{n}\sum_{t=0}^{n-1}|p_{d,\ell}(\omega^t)|^2 = \frac{1}{n} \sum_{t=0}^{n-1} (|A_{d,\ell,t}|^2 \pm \frac{1}{4}) =  (\sum_{t=0}^{n-1} |A_{d,\ell,t}|^2) \pm 1/4$
\item We have $|A_{d,\ell,t}|^2 = \bigo(H^2)$.
\item We sample $\bigo(H^4 \log n)$ of $|A_{d,\ell,t}|^2$ to estimate $\|p_{d,\ell}\|_2^2$ up to $\pm \frac{1}{2}$.
\item In total $\bigo(H^8 \log^2 n) = \bigo(k^8 L^8 \log^2 n)$ samples to compute such estimate.
\end{enumerate}

Applying to our tree-traversal, we get $\bigo(k^9 L^8 \log^3 n)$ complexity. Once we have the indices of non-zero coefficients, we extract the exact values:
$$\hat{a}_d = p_{d,\log n}(\omega^0) = \frac{1}{n} \sum_{i=0}^{n-1} a_{i}\omega^{di}$$
via sampling.

\paragraph{Second solution:}
Let $T$ be a support of $\hat{a}_i$, that is $i \in T$ iff $\hat{a}_i \not= 0$.
Using $\bigo(k^8 \log^2 n)$ samples we  estimate $\|p_{d,\ell}\|_2^2$ up to $\pm \frac{L^2}{16}$.

Using $\bigo(k^9 \log^3 n)$ samples we find $T' \subseteq T$ such that if $|\hat{a}_i| \ge L/4$ then $i \in T'$, and $\hat{a}'_i = \hat{a}_i \pm L/4$, where $\hat{a}'_i$ is defined only over $T'$.

Recurse on $(\hat{a}'_i - \hat{a}_i)$. Specifically, this sequence is also $k$-sparse, and max-value is $L/2$.

This makes it so the total complexity is $\bigo(k^9 \log^3 n \cdot \log_2 L)$. 

\section{Another algorithm, noisy case}
\subsection{$k=1$}

There is some heavy $\hat{a}_u$ such that $\sum_{u' \not = u} |\hat{a}_{u'}|^2 \le \varepsilon |\hat{a}_u|^2$, for some small constant $\varepsilon$.

Idea: extract $u$ bit-by-bit.
\paragraph{No noise:} If $u = 2v + b_0$ for $b_0 \in \{0,1\}$, then $a_{n/2} = \frac{1}{\sqrt{n}} \hat{a}_u\omega^{u n/2} =\frac{1}{\sqrt{n}}  \hat{a}_u \omega^{vn + b_0n/2} = \frac{1}{\sqrt{n}} \hat{a}_u \omega^{b_0n/2} = \frac{1}{\sqrt{n}} \hat{a}_u (-1)^{b_0}$.
Then, $a_0 = \frac{1}{\sqrt{n}} \hat{a}_u$. Thus we can use following test:

$$b_0 = 0 \quad\quad\text{iff}\quad\quad |a_0 - a_{n/2}| \le |a_0 + a_{n/2}|$$

How to test for older bits? Assume wlog that $b_0 = 0$, since if $b_0 = 1$, we can always consider signal $a'$ defined as $a'_j = a_j \cdot \omega^j$, where $\hat{a}'_j = \hat{a}_{j-1}$.
So $u = 4v' + 2b_1$, where $b_1 \in \{0,1\}$. We then observe that $a_{n/4} = \frac{1}{\sqrt{n}} \hat{a}_u (-1)^{b_1}$, so the test is then
$$b_1 = 0 \quad\quad\text{iff}\quad\quad |a_0 - a_{n/4}| \le |a_0 + a_{n/4}|$$

So we can proceed with all the bits in this manner.

\paragraph{Noisy case:}
Consider test for bit 0. If the noise is concentrated around $a_0$ and $a_{n/2}$, then such test fails. But we know that on average the noise is small. Thus we replace the test with a randomized one: pick $0 \le r < n$ at random, and test:
$$b_0 = 0 \quad\quad\text{iff}\quad\quad |a_{r} - a_{r+n/2}| \le |a_{r} + a_{r+n/2}|$$
(we can do many tests and pick majority vote)
and in general
$$b_{i-1} = 0 \quad\quad\text{iff}\quad\quad |a_r - a_{r+n/2^i}| \le |a_r + a_{r+n/2^i}|$$
of course assuming $b_0=b_1=..=b_{i-2} = 0$, and changing the signal accordingly.

Why does it work?

Let $\hat{a}'$ be the output. We show that with ppb at least $3/4$ there is $\|\hat{a} - \hat{a}'\|_2 \le \varepsilon \|\hat{a} - \hat{a}^{(1)}\|_2$, where $\hat{a}^{(1)}$ is the top coefficient, so $\hat{a} - \hat{a}^{(1)}$ is the noise.

We rewrite 
$$a_j = \frac{1}{\sqrt{n}}\hat{a}_u \omega^{uj} +  \frac{1}{\sqrt{n}}\sum_{u' \not= u} \hat{a}_{u'} \omega^{u'j} =   \frac{1}{\sqrt{n}}\hat{a}_u \omega^{uj} + \mu_j$$
so
$$a = F^{-1}\hat{a}^{(1)} + \mu$$

Looking at the error
$$\sum_{j=0}^{n-1} |\mu_j|^2 = \|\mu\|_2^2 = \|F^{-1}(\hat{a} - \hat{a}^{(1)})\|_2^2 = \| \hat{a} - \hat{a}^{(1)}\|_2^2 = \sum_{u' \not= u} |\hat{a}_{u'}|^2$$

Algorithm at single step compares $|a_{k} - a_{\ell}|$ vs $|a_{k} + a_\ell|$.
We have
$$a_k - a_\ell = \frac{1}{\sqrt{n}}\hat{a}_u(\omega^{uk} - \omega^{u\ell}) + (\mu_{k} - \mu_{\ell})$$
$$a_k + a_\ell = \frac{1}{\sqrt{n}}\hat{a}_u(\omega^{uk} + \omega^{u\ell}) + (\mu_{k} + \mu_{\ell})$$
We know that $|\omega^{uk} \pm \omega^{u\ell}| \in \{0,2\}$ so for the comparison to be done correctly, it is enough that $|\mu_k - \mu_\ell| + |\mu_k + \mu_\ell| \le 2 \frac{1}{\sqrt{n}} \hat{a}_u$, so $|\mu_k| + |\mu_\ell| \le \frac{1}{\sqrt{n}}\hat{a}_u$. 

We now have, since each index is picked with a random shift,
$$\E[|\mu_k|^2]  = \frac{1}{n} \sum_{j=0}^{n-1} |\mu_j|^2 = \frac{1}{n} \sum_{u' \not= u} |\hat{a}_{u'}|^2 \le \frac{1}{n} \varepsilon |\hat{a}_u|^2$$
so 
$$\Pr[|\mu_k| \le \frac{1}{2 \sqrt{n}} |\hat{a}_u|] \le  \frac{\frac1n \varepsilon |\hat{a}_u|^2}{\frac{1}{4n} |\hat{a}_u|^2} = \frac{\varepsilon}{4}$$
So picking $\varepsilon=1/2$ gives us by union bound $1/4$ ppb of success.

Now the trick is to amplify the ppb by repeating each test $\bigo(\log \log n)$ times and do the majority vote. This amplifies the ppb to $1/(4 \log n)$, so by union bound the whole procedure is ok with $3/4$ ppb.

\paragraph{Not covered:} how to extract the value of $\hat{a}_u$.
\end{document}


